\documentclass{article}

% Language setting
% Replace `english' with e.g. `spanish' to change the document language
\usepackage[english]{babel}

% Set page size and margins
% Replace `letterpaper' with `a4paper' for UK/EU standard size
\usepackage[letterpaper,top=2cm,bottom=2cm,left=3cm,right=3cm,marginparwidth=1.75cm]{geometry}

% Useful packages
\usepackage{amsmath}
\usepackage{graphicx}
\usepackage[colorlinks=true, allcolors=blue]{hyperref}

\title{The FOOT experiment: a first measurement of nuclear fragmentation cross section for hadrontherapy}
\author{Ubaldi G.%
	%\thanks{Electronic address: \texttt{giacomo.ubaldi@bo.infn.it}; Corresponding author}
	}
%\institute{University of Testing \and University of Diagnostics}	

	
\begin{document}
\maketitle



\section{Abstract}
%Hadrontherapy exploits beams of charged particles to irradiate the tumoral region of patients. According to how charged particles interact, the release of dose is mainly focused near the end of the path (Bragg peak) and this high selectivity involves killing cancer cells but sparing healthy tissues before and after the tumor. However,  nuclear interactions can also occur, causing release of dose even outside the Bragg peak due to the generation of fragments and these effects need to be considered.\\
%The FOOT (Fragmentation of Target) experiment was proposed with the aim of measuring double differential cross sections of nuclear interactions to overcome the lack of data in hadrontherapy energy range.\\
%A first analysis of data taken at GSI of a beam of $^{16}O$ against a target of polyethylene ($C_2H_4$) was carried out, obtaining the first evaluations of differential cross sections as a function of the fragment charge ($ 1 \leq Z \leq 8$) and angle direction ($ 1 ^{\circ} \leq \theta \leq 10 ^{\circ}$). \\
%After a short introduction about the role of the FOOT experiment in hadrontherapy and space radioprotection, the data analysis is described showing the first meaningful results. (1198 caratteri)\\ 

The FOOT (FragmentatiOn Of Target) experiment has been conceived with the main aim of measuring differential nuclear cross sections of target and beam fragments in the energy range of interest for hadrontherapy and space radioprotection, which suffers of lack of experimental results.\\
A first analysis of data taken at the GSI with a beam of $^{16}O$ at both 200 $MeV/n$ and 400 $MeV/n$ against two targets of polyethylene ($C_2H_4$) and carbon ($C$) will be presented, showing the first measurements of differential cross sections as a function of the fragment charge ($ 1 \leq Z \leq 8$) and angle ($ 1 ^{\circ} \leq \theta \leq 10 ^{\circ}$) and total energy distribution. \\
The analysis is performed taking full advantage of the current performances of the apparatus, with a particular focus on the charge estimation algorithm and on the optimization of the track reconstruction Kalman filter algorithm, both fundamental for the correct fragment identification and the measurements of its kinemastics. A pile-up reduction method will also be shown, specifically developed studying the data present in this analisys.

%and the reason for which the effects of the generated fragments need to be considered?)




\bibliographystyle{alpha}
\bibliography{sample}

\end{document}